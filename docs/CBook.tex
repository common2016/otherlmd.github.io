% Options for packages loaded elsewhere
\PassOptionsToPackage{unicode}{hyperref}
\PassOptionsToPackage{hyphens}{url}
%
\documentclass[
]{book}
\usepackage{lmodern}
\usepackage{amssymb,amsmath}
\usepackage{ifxetex,ifluatex}
\ifnum 0\ifxetex 1\fi\ifluatex 1\fi=0 % if pdftex
  \usepackage[T1]{fontenc}
  \usepackage[utf8]{inputenc}
  \usepackage{textcomp} % provide euro and other symbols
\else % if luatex or xetex
  \usepackage{unicode-math}
  \defaultfontfeatures{Scale=MatchLowercase}
  \defaultfontfeatures[\rmfamily]{Ligatures=TeX,Scale=1}
\fi
% Use upquote if available, for straight quotes in verbatim environments
\IfFileExists{upquote.sty}{\usepackage{upquote}}{}
\IfFileExists{microtype.sty}{% use microtype if available
  \usepackage[]{microtype}
  \UseMicrotypeSet[protrusion]{basicmath} % disable protrusion for tt fonts
}{}
\makeatletter
\@ifundefined{KOMAClassName}{% if non-KOMA class
  \IfFileExists{parskip.sty}{%
    \usepackage{parskip}
  }{% else
    \setlength{\parindent}{0pt}
    \setlength{\parskip}{6pt plus 2pt minus 1pt}}
}{% if KOMA class
  \KOMAoptions{parskip=half}}
\makeatother
\usepackage{xcolor}
\IfFileExists{xurl.sty}{\usepackage{xurl}}{} % add URL line breaks if available
\IfFileExists{bookmark.sty}{\usepackage{bookmark}}{\usepackage{hyperref}}
\hypersetup{
  pdftitle={R语言笔记},
  pdfauthor={云腾足下},
  hidelinks,
  pdfcreator={LaTeX via pandoc}}
\urlstyle{same} % disable monospaced font for URLs
\usepackage{color}
\usepackage{fancyvrb}
\newcommand{\VerbBar}{|}
\newcommand{\VERB}{\Verb[commandchars=\\\{\}]}
\DefineVerbatimEnvironment{Highlighting}{Verbatim}{commandchars=\\\{\}}
% Add ',fontsize=\small' for more characters per line
\usepackage{framed}
\definecolor{shadecolor}{RGB}{248,248,248}
\newenvironment{Shaded}{\begin{snugshade}}{\end{snugshade}}
\newcommand{\AlertTok}[1]{\textcolor[rgb]{0.94,0.16,0.16}{#1}}
\newcommand{\AnnotationTok}[1]{\textcolor[rgb]{0.56,0.35,0.01}{\textbf{\textit{#1}}}}
\newcommand{\AttributeTok}[1]{\textcolor[rgb]{0.77,0.63,0.00}{#1}}
\newcommand{\BaseNTok}[1]{\textcolor[rgb]{0.00,0.00,0.81}{#1}}
\newcommand{\BuiltInTok}[1]{#1}
\newcommand{\CharTok}[1]{\textcolor[rgb]{0.31,0.60,0.02}{#1}}
\newcommand{\CommentTok}[1]{\textcolor[rgb]{0.56,0.35,0.01}{\textit{#1}}}
\newcommand{\CommentVarTok}[1]{\textcolor[rgb]{0.56,0.35,0.01}{\textbf{\textit{#1}}}}
\newcommand{\ConstantTok}[1]{\textcolor[rgb]{0.00,0.00,0.00}{#1}}
\newcommand{\ControlFlowTok}[1]{\textcolor[rgb]{0.13,0.29,0.53}{\textbf{#1}}}
\newcommand{\DataTypeTok}[1]{\textcolor[rgb]{0.13,0.29,0.53}{#1}}
\newcommand{\DecValTok}[1]{\textcolor[rgb]{0.00,0.00,0.81}{#1}}
\newcommand{\DocumentationTok}[1]{\textcolor[rgb]{0.56,0.35,0.01}{\textbf{\textit{#1}}}}
\newcommand{\ErrorTok}[1]{\textcolor[rgb]{0.64,0.00,0.00}{\textbf{#1}}}
\newcommand{\ExtensionTok}[1]{#1}
\newcommand{\FloatTok}[1]{\textcolor[rgb]{0.00,0.00,0.81}{#1}}
\newcommand{\FunctionTok}[1]{\textcolor[rgb]{0.00,0.00,0.00}{#1}}
\newcommand{\ImportTok}[1]{#1}
\newcommand{\InformationTok}[1]{\textcolor[rgb]{0.56,0.35,0.01}{\textbf{\textit{#1}}}}
\newcommand{\KeywordTok}[1]{\textcolor[rgb]{0.13,0.29,0.53}{\textbf{#1}}}
\newcommand{\NormalTok}[1]{#1}
\newcommand{\OperatorTok}[1]{\textcolor[rgb]{0.81,0.36,0.00}{\textbf{#1}}}
\newcommand{\OtherTok}[1]{\textcolor[rgb]{0.56,0.35,0.01}{#1}}
\newcommand{\PreprocessorTok}[1]{\textcolor[rgb]{0.56,0.35,0.01}{\textit{#1}}}
\newcommand{\RegionMarkerTok}[1]{#1}
\newcommand{\SpecialCharTok}[1]{\textcolor[rgb]{0.00,0.00,0.00}{#1}}
\newcommand{\SpecialStringTok}[1]{\textcolor[rgb]{0.31,0.60,0.02}{#1}}
\newcommand{\StringTok}[1]{\textcolor[rgb]{0.31,0.60,0.02}{#1}}
\newcommand{\VariableTok}[1]{\textcolor[rgb]{0.00,0.00,0.00}{#1}}
\newcommand{\VerbatimStringTok}[1]{\textcolor[rgb]{0.31,0.60,0.02}{#1}}
\newcommand{\WarningTok}[1]{\textcolor[rgb]{0.56,0.35,0.01}{\textbf{\textit{#1}}}}
\usepackage{longtable,booktabs}
% Correct order of tables after \paragraph or \subparagraph
\usepackage{etoolbox}
\makeatletter
\patchcmd\longtable{\par}{\if@noskipsec\mbox{}\fi\par}{}{}
\makeatother
% Allow footnotes in longtable head/foot
\IfFileExists{footnotehyper.sty}{\usepackage{footnotehyper}}{\usepackage{footnote}}
\makesavenoteenv{longtable}
\usepackage{graphicx,grffile}
\makeatletter
\def\maxwidth{\ifdim\Gin@nat@width>\linewidth\linewidth\else\Gin@nat@width\fi}
\def\maxheight{\ifdim\Gin@nat@height>\textheight\textheight\else\Gin@nat@height\fi}
\makeatother
% Scale images if necessary, so that they will not overflow the page
% margins by default, and it is still possible to overwrite the defaults
% using explicit options in \includegraphics[width, height, ...]{}
\setkeys{Gin}{width=\maxwidth,height=\maxheight,keepaspectratio}
% Set default figure placement to htbp
\makeatletter
\def\fps@figure{htbp}
\makeatother
\setlength{\emergencystretch}{3em} % prevent overfull lines
\providecommand{\tightlist}{%
  \setlength{\itemsep}{0pt}\setlength{\parskip}{0pt}}
\setcounter{secnumdepth}{5}
\usepackage{ctex}

%\usepackage{xltxtra} % XeLaTeX的一些额外符号
% 设置中文字体
%\setCJKmainfont[BoldFont={黑体},ItalicFont={楷体}]{新宋体}

% 设置边距
\usepackage{geometry}
\geometry{%
  left=2.0cm, right=2.0cm, top=3.5cm, bottom=2.5cm} 

\usepackage{amsthm,mathrsfs}
\usepackage{booktabs}
\usepackage{longtable}
\makeatletter
\def\thm@space@setup{%
  \thm@preskip=8pt plus 2pt minus 4pt
  \thm@postskip=\thm@preskip
}
\makeatother
\usepackage[]{natbib}
\bibliographystyle{apalike}

\title{R语言笔记}
\author{云腾足下}
\date{2020年5月}

\begin{document}
\maketitle

{
\setcounter{tocdepth}{1}
\tableofcontents
}
\hypertarget{ux7b80ux4ecb}{%
\chapter*{简介}\label{ux7b80ux4ecb}}
\addcontentsline{toc}{chapter}{简介}

\begin{quote}
十步杀一人,千里不留行
事了拂衣去,不留功与名
\end{quote}

\hypertarget{PanelData}{%
\chapter{面板数据}\label{PanelData}}

\hypertarget{ux4e00ux53e5ux8bddtips}{%
\section{一句话Tips}\label{ux4e00ux53e5ux8bddtips}}

\begin{itemize}
\tightlist
\item
  \texttt{PSTR}:面板平滑转移模型。
\item
  \texttt{MSCMT}:多个结果变量的合成控制方法的包。
\item
  检查面板数据是否平衡:使用\texttt{table(PanelData{[},1:2{]})}或者\texttt{is.banance}。
\item
  \texttt{phtt}包,交互效应的面板模型,用的Bai (2009)的估计方法。It offers the possibility of analyzing panel data with large dimensions n and T and can be considered when the unobserved heterogeneity effects are time-varying.
\end{itemize}

\hypertarget{plmux5305}{%
\section{\texorpdfstring{\texttt{plm}包}{plm包}}\label{plmux5305}}

\begin{itemize}
\tightlist
\item
  包中的\texttt{vcovG}函数可以计算聚类标准误。一般这么用:
\end{itemize}

\begin{Shaded}
\begin{Highlighting}[]
\KeywordTok{summary}\NormalTok{(plm, }\DataTypeTok{vcov =} \KeywordTok{vcovG}\NormalTok{(plm, }\DataTypeTok{cluster =} \StringTok{'group'}\NormalTok{, }\DataTypeTok{inner =} \StringTok{'cluster'}\NormalTok{))}
\end{Highlighting}
\end{Shaded}

\begin{itemize}
\tightlist
\item
  包中的\texttt{fixef}函数可以返回个体截距项(\texttt{type\ =\ level})。
\item
  \texttt{updata(object,\ formula)}函数可以更新公式重新估计。
\end{itemize}

\hypertarget{ux52a8ux6001ux9762ux677fux9608ux503cux4f30ux8ba1rux8bedux8a00ux4e2dux6709ux4e00ux4e2aux5305dtpux5176ux4f30ux8ba1ux51fdux6570ux4e3a}{%
\section{\texorpdfstring{动态面板阈值估计:R语言中有一个包\texttt{dtp},其估计函数为:}{动态面板阈值估计:R语言中有一个包dtp,其估计函数为:}}\label{ux52a8ux6001ux9762ux677fux9608ux503cux4f30ux8ba1rux8bedux8a00ux4e2dux6709ux4e00ux4e2aux5305dtpux5176ux4f30ux8ba1ux51fdux6570ux4e3a}}

\begin{Shaded}
\begin{Highlighting}[]
\KeywordTok{data}\NormalTok{(Mena)}
\NormalTok{reg<-}\KeywordTok{dtp}\NormalTok{(GDPPC }\OperatorTok{~}\StringTok{ }\NormalTok{FDI}\OperatorTok{+}\NormalTok{OPEN}\OperatorTok{|}\NormalTok{INF}\OperatorTok{|}\NormalTok{INF,Mena,}\DataTypeTok{index=}\KeywordTok{c}\NormalTok{(}\StringTok{"pays"}\NormalTok{,}\StringTok{"ann"}\NormalTok{),}\DecValTok{4}\NormalTok{,}\DecValTok{2}\NormalTok{,}\FloatTok{0.95}\NormalTok{,}\FloatTok{0.8}\NormalTok{,}\DecValTok{1}\NormalTok{,}\DataTypeTok{graph =} \OtherTok{TRUE}\NormalTok{)}
\KeywordTok{summary}\NormalTok{(reg)}
\end{Highlighting}
\end{Shaded}

注意:
- 第一根\texttt{\textbar{}}前的变量是不依赖区制的变量,中间由\texttt{\textbar{}}夹住的变量是阈值变量,最后一个\texttt{\textbar{}}后面的变量是依赖区制的变量(好遗憾,貌似只允许一个这样的变量)。
- \texttt{initnum}参数指的是模型中的内生变量。在动态面板中,一般是因变量的滞后值,因此在数据框中滞后因变量,然后选好该滞后因变量所在列的数字即可。注意,这个数字是在剔除了id和year标识之后的列的序号。
- 数据上千以后,估计过程有点慢,耐心等待。
- 输出中包含一个\texttt{gamma}参数,我揣摩是截距项。

\hypertarget{TS}{%
\chapter{普通回归和时间序列序列}\label{TS}}

\hypertarget{ux4e00ux53e5ux8bddtips-1}{%
\section{一句话Tips}\label{ux4e00ux53e5ux8bddtips-1}}

\begin{itemize}
\tightlist
\item
  \texttt{confint}函数返回系数的置信区间
\item
  \texttt{bssm}拟合非线性卡尔曼滤波的包。\texttt{pomp},\texttt{KFAS}也是。\texttt{pomp}好像接口更简单些,第四节有一个非线性的例子。
\item
  \texttt{NlinTS}一个利用神经网络的格兰杰因果非线性检验。
\item
  \texttt{slider}: 在任何R数据类型上提供类型稳定的滚动窗口函数,并支持累积窗口和扩展窗口。
\item
  \texttt{testcorr}: 提供计算单变量时间序列中自相关显著性、双变量时间序列中互相关显著性、多变量序列中皮尔逊相关显著性和单变量序列i.i.d.特性的测试统计量的功能。\\
\item
  \texttt{apt}一个阈值协整包。
\item
  \texttt{fDMA}动态平均模型。卡尔曼滤波的贝叶斯模型平均。
\item
  \texttt{MuMIn}利用信息准则进行模型平均的包。
\item
  \texttt{MSBVAR}提供了贝叶斯框架下的马尔科夫转移VAR。\texttt{MSwM}是一个单方程(非单变量)的马尔科夫转移模型估计。
\item
  因子变虚拟变量:\texttt{model.matrix}可以生成回归所需要的矩阵,可以把因子变量变成虚拟变量。
\item
  \texttt{mfGARCH}包估计混频GARCH。
\item
  \texttt{TED::ur.za.fast}和\texttt{urca::ur.za}未知断点的单位根检验。
\item
  \texttt{mFilter}包有各种经济和金融常用的滤波,如HP,BK等滤波(好像没有更新了,官网包的镜像没有找到)。但是可以使用\texttt{FRAPO}包的\texttt{trdhp}函数来计算HP滤波 。
\item
  \texttt{svars}是一个数据驱动的结构VAR包。\texttt{vars}是一个VAR各种估计和诊断的标准包。\texttt{tsDyn}也有线性VAR和VECM的估计,其中它还允许包含外生变量。
\item
  \texttt{lmtest}有\texttt{grangertest()}做双变量格兰杰因果检验。\texttt{MTS::GrangerTest(regdata{[},-c(1,2){]},2,locInput\ =\ 1)}也可以,而且可以做多个变量是不是某个变量的格兰杰原因。\texttt{locInput}表示因变量是第几列。
\item
  \texttt{stats4}包提供了函数\texttt{mle}可以进行极大似然估计,还可以固定部分参数,优化其他参数,这其实是集中似然的思想。关键是它还返回方差协方差矩阵。语法如下,
\end{itemize}

\begin{Shaded}
\begin{Highlighting}[]
\KeywordTok{mle}\NormalTok{(minuslogl, }\DataTypeTok{start =} \KeywordTok{formals}\NormalTok{(minuslogl), }\DataTypeTok{method =} \StringTok{"BFGS"}\NormalTok{,}
    \DataTypeTok{fixed =} \KeywordTok{list}\NormalTok{(), nobs, ...) }\CommentTok{# 注意它的初值是一个list}
\end{Highlighting}
\end{Shaded}

\begin{itemize}
\tightlist
\item
  \texttt{dynlm::dynlm}包一个比\texttt{lm}更强大线性回归结构,优点有三:

  \begin{itemize}
  \tightlist
  \item
    可以使用差分、滞后等表述,如\texttt{d(y)\textasciitilde{}L(y,2)},可以直接添加趋势项\texttt{trend(y)}将使用\texttt{\$(1:n)/Freq\$}作为回归元。
  \item
    可以进行工具变量估计。
  \end{itemize}
\end{itemize}

但要注意,他的数据不是数据框,而是一个\texttt{ts}对象。
- \texttt{nardl}估计非线性协整分布滞后模型。
- \texttt{rugarch}:单变量garch建模。一个比\texttt{forcast}更好用的时序建模包。可以用\texttt{show}函数来返回一个丰富的结果,包括一些检验结果。
- \texttt{rmgarch}:多变量garch建模。包括dcc,adcc,gdcc等。
- \texttt{stats}包中的\texttt{ARMAtoMA}函数可以计算AR变成MA。\texttt{vars}包的\texttt{Phi}返回VAR的移动平均系数。
- \texttt{vars}包里面的\texttt{Phi}函数可以把VAR变成VMA。使用\texttt{summary}函数来摘要var的估计结果,会给粗特征根,残差相关矩阵等。
- \texttt{tsDyn}包的\texttt{VECM}函数比较好用,可以包括外生变量,可以选择OLS或Joson方法。这个包也是可以估计线性VAR的,主要是他的\texttt{lineVar}函数。\texttt{egcm}包是恩格尔格兰杰协整检验,这个检验在\texttt{urca}包里业可行。
- \texttt{TSA::periodogram}可以做谱分解。
- \texttt{bvarsv}时变参数var建模
- \texttt{nls}非线性最小二乘法函数
- \texttt{highfrequance}里面有不少意思的函数,包括\texttt{HAR}。

  \bibliography{mybib.bib}

\end{document}
